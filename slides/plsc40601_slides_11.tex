% !Rnw weave = Sweave
\documentclass[xcolor={dvipsnames}, handout]{beamer}


\RequirePackage{../assets/pres-template_MOW}


%--------------------------------------------------------------------------
% Specific to this document ---------------------------------------


%--------------------------------------------------------------------------
% \setbeamercovered{transparent}

%%%%%%%%%%%%%%%%%%%%%%%%%%%%%%%%%%%
\setlength{\tabcolsep}{1.3pt}
\title{PLSC 40601}
\subtitle{Week 1: Course orientation, potential outcomes framework.}
\date{Spring 2023}
\author{Molly Offer-Westort}
\institute{Department of Political Science, \\University of Chicago}


\usepackage{Sweave}
\begin{document}
\input{plsc40601_slides_11-concordance}

%-------------------------------------------------------------------------------%
\frame{\titlepage
\thispagestyle{empty}
}
%-------------------------------------------------------------------------------%
\begin{frame}{Overview}

This course:\pause
\begin{wideitemize}
\item Objectives. \pause
\item Course structure. \pause
\item Assignments. 
\end{wideitemize}
\vskip 0.5cm
\pause



Link to \href{https://github.com/UChicago-pol-methods/plsc-40601-CI-ML/blob/main/syllabus_PLSC40601.pdf}{syllabus}.

\end{frame}


%%%%%NOTE%%%%%
\note{
\scriptsize \singlespacing

\begin{wideitemize}
\item xxxx
\end{wideitemize}

}

%-------------------------------------------------------------------------------%
\begin{frame}{Final assignment}

\begin{wideitemize}
\item Possibility of tutorial. \pause
\begin{wideitemize}
\item Illustrate what problem the tool is addressing, what an example use case is. \pause
\item Still needs some amount of narrative contextualizing the problem. \pause
\item Posted on class github.\pause
\item Python, R, or Stata; \textbf{compiled report}. \pause
\item Limited support with debugging. 
\end{wideitemize}
\end{wideitemize}
\vskip 0.5cm
\pause


Link to \href{https://cyrussamii.com/?p=2907}{website}.
Link to \href{https://cdsamii.github.io/cds-demos/conformal/conformal-tutorial.html}{tutorial}.

\end{frame}


%%%%%NOTE%%%%%
\note{
\scriptsize \singlespacing

\begin{wideitemize}
\item xxxx
\end{wideitemize}

}

%-------------------------------------------------------------------------------%
\begin{frame}{How to read methods papers \citep{murphy1997read}}


Consider:\pause
\begin{wideitemize}
\item Who the authors are and where the article is published. \pause
\begin{wideitemize}
\item Econometrica? AER? Annual Review of Economics?
\item PNAS? Science? Nature?
\item JRSS B? JASA, Statistical Science?
\item SSRN, arXiv, PsyArXiv?
\item Machine learning conference proceedings?
\item Political science methods, biostatistics journal?
\end{wideitemize}
\end{wideitemize}

\end{frame}


%%%%%NOTE%%%%%
\note{
\scriptsize \singlespacing

\begin{wideitemize}
\item xxxx
\end{wideitemize}

}

%-------------------------------------------------------------------------------%
\begin{frame}{How to read methods papers}

Consider:\pause
\begin{wideitemize}
\item What is the problem/gap/contribution? \pause
\item What is the background/methodological context for the paper? \pause
\begin{wideitemize}
\item Refer to references where it may be more clearly stated; textbooks are often clearer than original articles for understanding a new methodology. \pause
\item Check out the bibliographic notes in a related section of \cite{hastie2009elements}, The Elements of Statistical Learning. \pause (If you don't have this book already, download it right now.) \pause
\end{wideitemize}
\item (This is the role of final paper.)
\end{wideitemize}

\end{frame}


%%%%%NOTE%%%%%
\note{
\scriptsize \singlespacing

\begin{wideitemize}
\item xxxx
\end{wideitemize}

}

%-------------------------------------------------------------------------------%
\begin{frame}{How to read methods papers}

Consider:\pause
\begin{wideitemize}
\item How would you explain this paper to someone else? \pause
\item What is a use case for the tools presented in the paper? \pause
\item (This is the role of paper presentation/discussion)
\end{wideitemize}

\end{frame}


%%%%%NOTE%%%%%
\note{
\scriptsize \singlespacing

\begin{wideitemize}
\item xxxx
\end{wideitemize}

}

%-------------------------------------------------------------------------------%
\begin{frame}{How to read methods papers}

Consider:\pause
\begin{wideitemize}
\item How would you use the presented method on data? \pause
\begin{wideitemize}
\item Many of the papers have associated packages. \pause
\end{wideitemize}
\item When using real data, what are limitations of the assumptions required by the method?\pause
\item (Tutorial?)
\end{wideitemize}

\end{frame}


%%%%%NOTE%%%%%
\note{
\scriptsize \singlespacing

\begin{wideitemize}
\item xxxx
\end{wideitemize}

}

%-------------------------------------------------------------------------------%
\begin{frame}{How to read methods papers}

Consider:\pause
\begin{wideitemize}
\item Annotated bibliography?
\end{wideitemize}

\end{frame}


%%%%%NOTE%%%%%
\note{
\scriptsize \singlespacing

\begin{wideitemize}
\item xxxx
\end{wideitemize}

}

%-------------------------------------------------------------------------------%
\begin{frame}{How to read methods papers}

References for notation:\pause
\begin{wideitemize}
\item \cite{hastie2009elements}
\item Glossary of \cite{aronow2019foundations} Foundations of Agnostic Statistics
\item Other suggestions?
\end{wideitemize}

\end{frame}


%%%%%NOTE%%%%%
\note{
\scriptsize \singlespacing

\begin{wideitemize}
\item xxxx
\end{wideitemize}

}

%-------------------------------------------------------------------------------%

\begin{transitionframe}
\centering

\LARGE \textcolor{white}{Potential outcomes framework.}

\end{transitionframe}
%-------------------------------------------------------------------------------%

\begin{frame}{Statistical setup.}

\begin{wideitemize}
\item Population. \pause
\item Sample, units indexed $i = 1, \dots, N$.\pause
\item Observed outcome, $Y_i\in \RR$;\pause
\item Treatments, $W_i \in \{0, \dots, K\}$; \pause (or $Z_i$, or $D_i$, $A_i$,\dots)\pause
\begin{itemize}
\item What do we mean by treatments? \pause
\item ``No causation without manipulation.'' \citep{holland1986}\pause
\end{itemize}
\item Covariates, $X_i \in \RR^p$.
\end{wideitemize}

\end{frame}


%%%%%NOTE%%%%%
\note{
\scriptsize \singlespacing

\begin{wideitemize}
\item xxxx
\end{wideitemize}

}

%-------------------------------------------------------------------------------%
\begin{frame}{Statistical setup.}

\begin{wideitemize}
\item Potential outcomes framework: $Y_i(w)$ represents the potential outcome for respondent $i$ under treatment $w$. \pause $\Y_i = (Y_i(0), \dots, Y_i(K))$.\pause
\item $Y_i(1)$ is the outcome we would see only when individual $i$ receives treatment $1$. \pause
\item Alternatively, (for binary treatment, $W_i \in\{0,1\}$) 
\begin{itemize}
\item $Y_i^1, Y_i^0$; 
\item $Y_i^{w=1},Y_i^{w=0}$; 
\item $Y_{i1}, Y_{i0}$; 
\item $Y|\textrm{do}(W = 1), Y|\textrm{do}(W = 0)$
\end{itemize}
\end{wideitemize}

\end{frame}


%%%%%NOTE%%%%%
\note{
\scriptsize \singlespacing

\begin{wideitemize}
\item xxxx
\end{wideitemize}

}

%-------------------------------------------------------------------------------%
\begin{frame}{(Some) Causal estimands}
\vskip 1 ex
\pause
What is an estimand? \pause What makes an estimand \textit{causal}? \pause Counterfactual comparison.  
\vskip 2 ex
\pause
\begin{wideitemize}
\item Individual treatment effect: \[\tau_i = Y_i(1) - Y_i(0)\]\pause
\item Average treatment effect (ATE): \[\tau = \E[\tau_i]\] \pause 
\begin{itemize}
\item Expectation over what?\pause
\item[$\rightarrow$] The population we previously defined.
\end{itemize}
\end{wideitemize}

\end{frame}


%%%%%NOTE%%%%%
\note{
\scriptsize \singlespacing

\begin{wideitemize}
\item xxxx
\end{wideitemize}

}

%-------------------------------------------------------------------------------%
\begin{frame}{(Some) Causal estimands}

\begin{wideitemize}
\item Average treatment effect on the treated (ATT): \[\tau_{ATT} = \E[\tau_i | W_i = 1]\] \pause
\begin{itemize}
\item How is this different from the ATE?
\item Why might we care about this?
\end{itemize}\pause
\item Conditional average treatment effect (CATE): \[\tau_{CATE} = \E[\tau_i | X_i = x]\] \pause
\end{wideitemize}

\centering
Many others!

\end{frame}


%%%%%NOTE%%%%%
\note{
\scriptsize \singlespacing

\begin{wideitemize}
\item xxxx
\end{wideitemize}

}

%-------------------------------------------------------------------------------%

\backupbegin
%-------------------------------------------------------------------------------%

\begin{frame}[allowframebreaks]{References}
    \bibliographystyle{apalike}
    \bibliography{../assets/PLSC40601}
\end{frame}
%-------------------------------------------------------------------------------%
\backupend
\end{document}
%
%-------------------------------------------------------------------------------%

%%% [[TEMPLATE]] %%%
\begin{frame}{Frametitle}

\begin{wideitemize}
\item xxx
\end{wideitemize}

\end{frame}


%%%%%NOTE%%%%%
\note{
\scriptsize \singlespacing

\begin{wideitemize}
\item xxxx
\end{wideitemize}

}

%-------------------------------------------------------------------------------%
